\documentclass{article}

\usepackage{arxiv}

\usepackage[utf8]{inputenc} % allow utf-8 input
\usepackage[T1]{fontenc}    % use 8-bit T1 fonts
\usepackage{lmodern}        % https://github.com/rstudio/rticles/issues/343
\usepackage{hyperref}       % hyperlinks
\usepackage{url}            % simple URL typesetting
\usepackage{booktabs}       % professional-quality tables
\usepackage{amsfonts}       % blackboard math symbols
\usepackage{nicefrac}       % compact symbols for 1/2, etc.
\usepackage{microtype}      % microtypography
\usepackage{lipsum}
\usepackage{graphicx}

\title{Capitalized Title Here}

\author{
    Emi Tanaka
    \thanks{emitanaka.org}
   \\
    Department of Econometrics and Business Statistics \\
    Monash University \\
  Clayton, VIC 3800 \\
  \texttt{\href{mailto:emi.tanaka@monash.edu}{\nolinkurl{emi.tanaka@monash.edu}}} \\
  }

\usepackage{color}
\usepackage{fancyvrb}
\newcommand{\VerbBar}{|}
\newcommand{\VERB}{\Verb[commandchars=\\\{\}]}
\DefineVerbatimEnvironment{Highlighting}{Verbatim}{commandchars=\\\{\}}
% Add ',fontsize=\small' for more characters per line
\usepackage{framed}
\definecolor{shadecolor}{RGB}{248,248,248}
\newenvironment{Shaded}{\begin{snugshade}}{\end{snugshade}}
\newcommand{\AlertTok}[1]{\textcolor[rgb]{0.94,0.16,0.16}{#1}}
\newcommand{\AnnotationTok}[1]{\textcolor[rgb]{0.56,0.35,0.01}{\textbf{\textit{#1}}}}
\newcommand{\AttributeTok}[1]{\textcolor[rgb]{0.77,0.63,0.00}{#1}}
\newcommand{\BaseNTok}[1]{\textcolor[rgb]{0.00,0.00,0.81}{#1}}
\newcommand{\BuiltInTok}[1]{#1}
\newcommand{\CharTok}[1]{\textcolor[rgb]{0.31,0.60,0.02}{#1}}
\newcommand{\CommentTok}[1]{\textcolor[rgb]{0.56,0.35,0.01}{\textit{#1}}}
\newcommand{\CommentVarTok}[1]{\textcolor[rgb]{0.56,0.35,0.01}{\textbf{\textit{#1}}}}
\newcommand{\ConstantTok}[1]{\textcolor[rgb]{0.00,0.00,0.00}{#1}}
\newcommand{\ControlFlowTok}[1]{\textcolor[rgb]{0.13,0.29,0.53}{\textbf{#1}}}
\newcommand{\DataTypeTok}[1]{\textcolor[rgb]{0.13,0.29,0.53}{#1}}
\newcommand{\DecValTok}[1]{\textcolor[rgb]{0.00,0.00,0.81}{#1}}
\newcommand{\DocumentationTok}[1]{\textcolor[rgb]{0.56,0.35,0.01}{\textbf{\textit{#1}}}}
\newcommand{\ErrorTok}[1]{\textcolor[rgb]{0.64,0.00,0.00}{\textbf{#1}}}
\newcommand{\ExtensionTok}[1]{#1}
\newcommand{\FloatTok}[1]{\textcolor[rgb]{0.00,0.00,0.81}{#1}}
\newcommand{\FunctionTok}[1]{\textcolor[rgb]{0.00,0.00,0.00}{#1}}
\newcommand{\ImportTok}[1]{#1}
\newcommand{\InformationTok}[1]{\textcolor[rgb]{0.56,0.35,0.01}{\textbf{\textit{#1}}}}
\newcommand{\KeywordTok}[1]{\textcolor[rgb]{0.13,0.29,0.53}{\textbf{#1}}}
\newcommand{\NormalTok}[1]{#1}
\newcommand{\OperatorTok}[1]{\textcolor[rgb]{0.81,0.36,0.00}{\textbf{#1}}}
\newcommand{\OtherTok}[1]{\textcolor[rgb]{0.56,0.35,0.01}{#1}}
\newcommand{\PreprocessorTok}[1]{\textcolor[rgb]{0.56,0.35,0.01}{\textit{#1}}}
\newcommand{\RegionMarkerTok}[1]{#1}
\newcommand{\SpecialCharTok}[1]{\textcolor[rgb]{0.00,0.00,0.00}{#1}}
\newcommand{\SpecialStringTok}[1]{\textcolor[rgb]{0.31,0.60,0.02}{#1}}
\newcommand{\StringTok}[1]{\textcolor[rgb]{0.31,0.60,0.02}{#1}}
\newcommand{\VariableTok}[1]{\textcolor[rgb]{0.00,0.00,0.00}{#1}}
\newcommand{\VerbatimStringTok}[1]{\textcolor[rgb]{0.31,0.60,0.02}{#1}}
\newcommand{\WarningTok}[1]{\textcolor[rgb]{0.56,0.35,0.01}{\textbf{\textit{#1}}}}

% Pandoc citation processing



\begin{document}
\maketitle

\def\tightlist{}


\begin{abstract}
An abstract of less than 150 words.
\end{abstract}


\hypertarget{introduction}{%
\subsection{Introduction}\label{introduction}}

Introductory section which may include references in parentheses
({\textbf{???}}), or cite a reference such as ({\textbf{???}}) in the
text.

\hypertarget{section-title-in-sentence-case}{%
\subsection{Section title in sentence
case}\label{section-title-in-sentence-case}}

This section may contain a figure such as Figure \ref{fig:Rlogo}.

\begin{figure}[htbp]

{\centering \includegraphics[width=2in]{/Users/etan0038/Dropbox/projects/paper-review-DoE-pkgs/Rlogo} 

}

\caption{The logo of R.}\label{fig:Rlogo}
\end{figure}

\hypertarget{another-section}{%
\subsection{Another section}\label{another-section}}

There will likely be several sections, perhaps including code snippets,
such as:

\begin{Shaded}
\begin{Highlighting}[]
\NormalTok{x \textless{}{-}}\StringTok{ }\DecValTok{1}\OperatorTok{:}\DecValTok{10}
\KeywordTok{plot}\NormalTok{(x)}
\end{Highlighting}
\end{Shaded}

\includegraphics{arxiv_files/figure-latex/unnamed-chunk-3-1.pdf}

\hypertarget{summary}{%
\subsection{Summary}\label{summary}}

This file is only a basic article template. For full details of
\emph{The R Journal} style and information on how to prepare your
article for submission, see the
\href{https://journal.r-project.org/share/author-guide.pdf}{Instructions
for Authors}.

\hypertarget{about-this-format-and-the-r-journal-requirements}{%
\subsubsection{About this format and the R Journal
requirements}\label{about-this-format-and-the-r-journal-requirements}}

\texttt{rticles::rjournal\_article} will help you build the correct
files requirements:

\begin{itemize}
\tightlist
\item
  A R file will be generated automatically using \texttt{knitr::purl} -
  see \url{https://bookdown.org/yihui/rmarkdown-cookbook/purl.html} for
  more information.
\item
  A tex file will be generated from this Rmd file and correctly included
  in \texttt{RJwapper.tex} as expected to build \texttt{RJwrapper.pdf}.
\item
  All figure files will be kept in the default rmarkdown
  \texttt{*\_files} folder. This happens because
  \texttt{keep\_tex\ =\ TRUE} by default in
  \texttt{rticles::rjournal\_article}
\item
  Only the bib filename is to modifed. An example bib file is included
  in the template (\texttt{RJreferences.bib}) and you will have to name
  your bib file as the tex, R, and pdf files.
\end{itemize}

\bibliographystyle{unsrt}
\bibliography{references.bib}


\end{document}
